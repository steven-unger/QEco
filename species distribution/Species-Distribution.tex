\PassOptionsToPackage{unicode=true}{hyperref} % options for packages loaded elsewhere
\PassOptionsToPackage{hyphens}{url}
%
\documentclass[]{article}
\usepackage{lmodern}
\usepackage{amssymb,amsmath}
\usepackage{ifxetex,ifluatex}
\usepackage{fixltx2e} % provides \textsubscript
\ifnum 0\ifxetex 1\fi\ifluatex 1\fi=0 % if pdftex
  \usepackage[T1]{fontenc}
  \usepackage[utf8]{inputenc}
  \usepackage{textcomp} % provides euro and other symbols
\else % if luatex or xelatex
  \usepackage{unicode-math}
  \defaultfontfeatures{Ligatures=TeX,Scale=MatchLowercase}
\fi
% use upquote if available, for straight quotes in verbatim environments
\IfFileExists{upquote.sty}{\usepackage{upquote}}{}
% use microtype if available
\IfFileExists{microtype.sty}{%
\usepackage[]{microtype}
\UseMicrotypeSet[protrusion]{basicmath} % disable protrusion for tt fonts
}{}
\IfFileExists{parskip.sty}{%
\usepackage{parskip}
}{% else
\setlength{\parindent}{0pt}
\setlength{\parskip}{6pt plus 2pt minus 1pt}
}
\usepackage{hyperref}
\hypersetup{
            pdftitle={Species distribution},
            pdfauthor={Steven Unger},
            pdfborder={0 0 0},
            breaklinks=true}
\urlstyle{same}  % don't use monospace font for urls
\usepackage[margin=1in]{geometry}
\usepackage{longtable,booktabs}
% Fix footnotes in tables (requires footnote package)
\IfFileExists{footnote.sty}{\usepackage{footnote}\makesavenoteenv{longtable}}{}
\usepackage{graphicx,grffile}
\makeatletter
\def\maxwidth{\ifdim\Gin@nat@width>\linewidth\linewidth\else\Gin@nat@width\fi}
\def\maxheight{\ifdim\Gin@nat@height>\textheight\textheight\else\Gin@nat@height\fi}
\makeatother
% Scale images if necessary, so that they will not overflow the page
% margins by default, and it is still possible to overwrite the defaults
% using explicit options in \includegraphics[width, height, ...]{}
\setkeys{Gin}{width=\maxwidth,height=\maxheight,keepaspectratio}
\setlength{\emergencystretch}{3em}  % prevent overfull lines
\providecommand{\tightlist}{%
  \setlength{\itemsep}{0pt}\setlength{\parskip}{0pt}}
\setcounter{secnumdepth}{0}
% Redefines (sub)paragraphs to behave more like sections
\ifx\paragraph\undefined\else
\let\oldparagraph\paragraph
\renewcommand{\paragraph}[1]{\oldparagraph{#1}\mbox{}}
\fi
\ifx\subparagraph\undefined\else
\let\oldsubparagraph\subparagraph
\renewcommand{\subparagraph}[1]{\oldsubparagraph{#1}\mbox{}}
\fi

% set default figure placement to htbp
\makeatletter
\def\fps@figure{htbp}
\makeatother


\title{Species distribution}
\author{Steven Unger}
\date{1/31/2020}

\begin{document}
\maketitle

\hypertarget{methods}{%
\section{Methods}\label{methods}}

\hypertarget{species-information}{%
\subsection{Species Information}\label{species-information}}

The Saguaro cactus (\emph{Carnegiea gigantea}) is a columnar cactus
endemic to the Sonoran Desert (western Sonora Desert and southern
Arizona). It is considered a keystone species because of how many
species rely on all the different life stages of the plant for food,
nesting, shade, etc (Drezner, 2014). There is increasing fears of
population collapse because of invasive grass species which are
encouraging a fire regime which Saguaro cacti did not evolve to
experience. As the climate warms, and grasses invade, saguaro cacti
populations are expected to decline (Hauser, 1993),(Marshall \emph{et
al}, 2012),(Schiermeier, 2005). Thus, it is critically important to
understand the species distribution to inform and direct conservation
and restoration efforts.

\begin{figure}
\centering
\includegraphics[width=2.08333in,height=\textheight]{/Users/Lee/Desktop/QEco/species distribution/Saguaro cactus.jpg}
\caption{Saguaro Cacrtus (\emph{Carnegiea gigantea})}
\end{figure}

\hypertarget{statistical-analysis}{%
\subsection{Statistical Analysis}\label{statistical-analysis}}

BIOCLIM is an algorithm is a very useful modeling method used for
species distribution, especially to teach newcomers. Despite newer
models performing better than BIOCLIM, the simplicity and ease of use
allow students to hone their skills and grasp otherwise complex
modelling. The algorith relies on constraining the similarity of
locations to the mean if they are near either tail of the distribution.
A transformation is performed by using the value obtained via using the
minimum tail, subtracting by 1 and multiplying by 2 to ensure a value
between 0 and 1. This is done to enable easier analysis interpretation.

\begin{figure}
\centering
\includegraphics{/Users/Lee/Desktop/QEco/species distribution/sagura dist.jpeg}
\caption{Figure 1: Saguaro distribution}
\end{figure}

\begin{longtable}[]{@{}ll@{}}
\caption{\textbf{Dependent species}}\tabularnewline
\toprule
Common.name & Use\tabularnewline
\midrule
\endfirsthead
\toprule
Common.name & Use\tabularnewline
\midrule
\endhead
Gila woodpeckers & Nests\tabularnewline
House Finches & Nests\tabularnewline
Gilded Flickers & Nests\tabularnewline
Purple Martins & Nests\tabularnewline
Elf owls & Secondary nests\tabularnewline
Tyrant Flycatchers & Secondary nests\tabularnewline
Wrens & Secondary nests\tabularnewline
White-wing dove & Eat fruits\tabularnewline
Ants & Eat fruits\tabularnewline
Honey bee & Eats nectar\tabularnewline
Flies & Eats decomposing tissue\tabularnewline
Beetles & Eats decomposing tissue\tabularnewline
Desert Iguana & eats flowers\tabularnewline
Sonora harvest mouse & eats seedlings\tabularnewline
Desert kangaroo rat & eats flowers and seedlings\tabularnewline
Gray Fox & eats fruit\tabularnewline
Jack rabbit & fruit\tabularnewline
Vulture & Roosting perch\tabularnewline
Humans & eat fruit and seeds, water vessel , use as feed\tabularnewline
\bottomrule
\end{longtable}

Table 1: List of some common species which depend on Saguaro cacti and
what it is that they use them for. Not an exhuastive list. Information
pulled from (Drezner, 2014), except human uses.

\hypertarget{discussion}{%
\section{\texorpdfstring{\textbf{Discussion}}{Discussion}}\label{discussion}}

A model showing the distribution of Saguaro cactus (\emph{Carnegiea
gigantea}) was created in R studio by using occurance data (GPS points
of where the species has been recorded) and environmental data (19
abiotic factors). The model used occurance points as well as
pseudo-absence points coupled with BIOCLIM data which factored 19
abiotic variables including various temperature and precipitation
parameters (max, min, mean, etc) into the model to not only map the
occurance, but also predict where other individuals are likely to exist.
This mapping distribution could be very useful to managment officials
and researchers who continue to monitor the effects of invasives, as
well as monitoring the health of the Sonoran desert by using the
keystone species Saguaro cactus as a proxy. While various methods have
been developed to more accurately map species distribution, this method
is a great starting point to teach young scientist how to code and
analysis species distributions.

\hypertarget{works-cited}{%
\subsubsection{Works Cited}\label{works-cited}}

Drezner, Taly Dawn (2014-06-01). ``The keystone saguaro (Carnegiea
gigantea, Cactaceae): a review of its ecology, associations,
reproduction, limits, and demographics''. Plant Ecology. 215 (6):
581--595.

Hauser, A. Scott (1993). ``Pennisetum ciliare''. US Forest Service Fire
Effects Information System. U.S. Department of Agriculture, US Forest
Service, Rocky Mountain Research Station, Fire Sciences Laboratory
(Producer).

Marshall, V. M.; Lewis, M. M.; Ostendorf, B. (2012-03-01). ``Buffel
grass (Cenchrus ciliaris) as an invader and threat to biodiversity in
arid environments: A review''. Journal of Arid Environments. 78: 1--12.

Schiermeier, Quirin (2005-06-01). ``Pall hangs over desert's future as
alien weeds fuel wildfires''. Nature. 435 (7043): 724.

\end{document}
