\PassOptionsToPackage{unicode=true}{hyperref} % options for packages loaded elsewhere
\PassOptionsToPackage{hyphens}{url}
%
\documentclass[]{article}
\usepackage{lmodern}
\usepackage{amssymb,amsmath}
\usepackage{ifxetex,ifluatex}
\usepackage{fixltx2e} % provides \textsubscript
\ifnum 0\ifxetex 1\fi\ifluatex 1\fi=0 % if pdftex
  \usepackage[T1]{fontenc}
  \usepackage[utf8]{inputenc}
  \usepackage{textcomp} % provides euro and other symbols
\else % if luatex or xelatex
  \usepackage{unicode-math}
  \defaultfontfeatures{Ligatures=TeX,Scale=MatchLowercase}
\fi
% use upquote if available, for straight quotes in verbatim environments
\IfFileExists{upquote.sty}{\usepackage{upquote}}{}
% use microtype if available
\IfFileExists{microtype.sty}{%
\usepackage[]{microtype}
\UseMicrotypeSet[protrusion]{basicmath} % disable protrusion for tt fonts
}{}
\IfFileExists{parskip.sty}{%
\usepackage{parskip}
}{% else
\setlength{\parindent}{0pt}
\setlength{\parskip}{6pt plus 2pt minus 1pt}
}
\usepackage{hyperref}
\hypersetup{
            pdftitle={Workshop 3\_Timeseries Models},
            pdfauthor={Steven Unger},
            pdfborder={0 0 0},
            breaklinks=true}
\urlstyle{same}  % don't use monospace font for urls
\usepackage[margin=1in]{geometry}
\usepackage{graphicx,grffile}
\makeatletter
\def\maxwidth{\ifdim\Gin@nat@width>\linewidth\linewidth\else\Gin@nat@width\fi}
\def\maxheight{\ifdim\Gin@nat@height>\textheight\textheight\else\Gin@nat@height\fi}
\makeatother
% Scale images if necessary, so that they will not overflow the page
% margins by default, and it is still possible to overwrite the defaults
% using explicit options in \includegraphics[width, height, ...]{}
\setkeys{Gin}{width=\maxwidth,height=\maxheight,keepaspectratio}
\setlength{\emergencystretch}{3em}  % prevent overfull lines
\providecommand{\tightlist}{%
  \setlength{\itemsep}{0pt}\setlength{\parskip}{0pt}}
\setcounter{secnumdepth}{0}
% Redefines (sub)paragraphs to behave more like sections
\ifx\paragraph\undefined\else
\let\oldparagraph\paragraph
\renewcommand{\paragraph}[1]{\oldparagraph{#1}\mbox{}}
\fi
\ifx\subparagraph\undefined\else
\let\oldsubparagraph\subparagraph
\renewcommand{\subparagraph}[1]{\oldsubparagraph{#1}\mbox{}}
\fi

% set default figure placement to htbp
\makeatletter
\def\fps@figure{htbp}
\makeatother


\title{Workshop 3\_Timeseries Models}
\author{Steven Unger}
\date{1/24/2020}

\begin{document}
\maketitle

\hypertarget{objectives}{%
\section{Objectives}\label{objectives}}

The primary objectives of this analysis is to apply the Autoregressive
Integrated Moving Average method (ARIMA) to a eddy covariance derived
time-series dataset and when/how to account for seasonal and temporal
trends which violate independence assumptions. Specifically i will be
investigating the role salinity plays in exlpaining the observed trends.

\hypertarget{methods}{%
\section{Methods}\label{methods}}

\hypertarget{site-information}{%
\subsection{Site Information}\label{site-information}}

The dataset used in this analysis originates from an eddy-covariance
tower the LTER site TS/Ph-7 in the Florida Everglades National Park
(Figure 1). This flux tower collects continues data on numerous
ecological parameters such as NEE, PAR, air temp, water temp, and
salinity (Figure 2).

\begin{figure}
\centering
\includegraphics[width=1.77083in,height=\textheight]{/Users/Lee/Desktop/QEco/time series/map of site.png}
\caption{Figure 1: Map of tower site}
\end{figure}

\begin{figure}
\centering
\includegraphics[width=1.77083in,height=\textheight]{/Users/Lee/Desktop/QEco/time series/Site picture.png}
\caption{Figure 2: Eddy covariance tower Site in Mangrove habitat}
\end{figure}

\hypertarget{statistical-analysis}{%
\subsection{Statistical Analysis}\label{statistical-analysis}}

Here we will be guided via
\href{https://fiu.instructure.com/courses/56707/pages/time-series-analysis}{Dr.~Malone's
Guide} (Malone, 2020). The statistical analysis of the time-series
dataset we employed is the Box-Jenkins method where we fit an ARIMA
model to time-series data. First, we plot out the timeseries data to
visualize any outliers or oddities, and then remove them if necessary.
Next we decompose the observed timeseries into a seasonal component
(takes into account calender fluctuations), trends (takes into account
trends observed not linked to seasonality), and whatever is not
exlpained is the residuals. Once deconstructed, we must test if the
timeseries is stationary (required when using ARIMA). IF it is, we
continue to test the autocorrelations which will identify our model
parameters as p (lag order),d (the degree of differencing), and q (the
moving average) to use in the fitting of the ARIMA model. Finally, we
ensure that the proper order of parameters was used so that we do not
have significant autocorrelations present. For reference to help with
ARIMA models, click here:
\href{http://webdoc.sub.gwdg.de/ebook/serien/e/monash_univ/wp6-07.pdf}{Automatic
time series forecasting: the forecast package for R} (Hyndman and
Khandakar, 2007).

\hypertarget{results}{%
\section{Results}\label{results}}

Through applying the ARIMA model, we first needed to decompose the time
series to decipher the seasonal trends, overarching trends, and the
residuals as shown in Figure 3 and then test for stationarity (a
requirement to run ARIMA models). After testing autocorrelations, we
compared our models using AIC, which showed that arima.nee4 model
performed better than our other arima runs as the lower value indicates
that the model accounted for more variation (see Table 1). Figure 4
shows the best fit ARIMA model overlaid in red on the NEE time series.

\begin{figure}
\centering
\includegraphics{/Users/Lee/Desktop/QEco/time series/Rplot-decomp.jpeg}
\caption{Figure 3: Decomposition of multiplicative time series of
salinity}
\end{figure}

\begin{figure}
\centering
\includegraphics{/Users/Lee/Desktop/QEco/time series/Rplot.png}
\caption{Figure 4: NEE with overlaid ARIMA model in Red}
\end{figure}

\begin{figure}
\centering
\includegraphics[width=1.30208in,height=\textheight]{/Users/Lee/Desktop/QEco/time series/Table1Done.png}
\caption{Table 1: AIC Results}
\end{figure}

\hypertarget{discussion}{%
\section{Discussion}\label{discussion}}

The Goal of this exersize was to apply ARIMA models to time series data,
and find what parameter best explains the patterns we observe, after we
factor out seasonal trends. We specifically tested salinity, and we
found that salinity does somewhat explain the pattern we observe,
however, I was unable to test air temp and water temp through all of the
protocols because of various errors. That being said, I suspect that
precipitation is largely responsible for the decrease in salinity which
is coupled with an increase in NEE because as the wet season approaches,
salinity in mangrove habitats would likely decrease because of the
larger proportion of freshwater. Conversely, I would expect that in the
dry season, more freshwater would evaporate leaving behind higher
salinity water. Finally, during times of high salinity, it may be harder
for the plants to photosynthesize because of the metabolic cost of
dealing with excess salts. This would require energy to exude, and would
overall decrease NEE. If you compare the salinity graph with the NEE
graph, you can see that NEE increases as Salinity decreases and
vice-versa.

Note to instructor: I would like to go over with you how to look at
other variables (tair, twater) and why changing the size of the graphs
affected the actual data it was showing me (when i was looking at
autocorrelations). I will be using this method for my proposal, but i
want to continue working on this dataset to truely understand how to run
my own models.

\hypertarget{works-cited}{%
\subsection{Works Cited}\label{works-cited}}

Hyndman, R. J., \& Khandakar, Y. (2007). Automatic time series for
forecasting: the forecast package for R (No.~6/07). Clayton VIC,
Australia: Monash University, Department of Econometrics and Business
Statistics.

Malone, Sparkle (2020). Time Series: Autoregressive Integrated Moving
Average - ARIMA Workshop, Florida International University, Department
of Biological Sciences

\end{document}
